\documentclass[conference]{IEEEtran}
\IEEEoverridecommandlockouts
% The preceding line is only needed to identify funding in the first footnote. If that is unneeded, please comment it out.

\usepackage[brazilian]{babel}
\usepackage[utf8]{inputenc}
\usepackage[T1]{fontenc}

\usepackage[T1]{fontenc} 
\usepackage[utf8]{inputenc}

\usepackage{cite}
\usepackage{amsmath,amssymb,amsfonts}
\usepackage{algorithmic}
\usepackage{graphicx}
\usepackage{textcomp}
\usepackage{xcolor}
\def\BibTeX{{\rm B\kern-.05em{\sc i\kern-.025em b}\kern-.08em
    T\kern-.1667em\lower.7ex\hbox{E}\kern-.125emX}}

\begin{document}

\title{Classificação de conteúdo áudio-textual usando rede neural convolucional Max Poolling no âmbito pericial}

\author{\IEEEauthorblockN{ Joberth Rogers Tavares Costa }
\IEEEauthorblockA{
\textit{Universidade de Brasília}\\
\textit{dept. FGA} \\
Gama, Distrito Federal\\
joberth.rogers18@gmail.com}
\and
\IEEEauthorblockN{ João Paulo Claudino de Sousa }
\IEEEauthorblockA{
\textit{Sessão de Computação Forense}\\
\textit{PCDF} \\
Brasília, Brasil\\
e-mail: jpclaudino@gmail.com}

}

\maketitle

\begin{abstract}
The use of machine learning algorithms is one of the most talked about topics in recent years in the world of information technology. The effectiveness of these types of algorithms for learning from common patterns found in the datasets that are used to train the model that will later classify the various objects through training is one reason for the constant use of this technology in various projects. Due to the long time taken to identify suspicious audios in the context of expert drug trafficking, the use of deep learning convolutional neural network algorithms (one of the branches of machine learning) shows great effectiveness in identifying these media through specific expression training where it involves suspicious content.    
\end{abstract}

\section{Resumo}

O uso de algoritmos de aprendizados de máquina é um dos temas mais falados nos últimos anos no mundo da Tecnologia da informação. A eficácia desses tipos de algoritmos para aprender com padrões comuns achado nos conjuntos de dados que são usados para treinar o modelo que  posteriormente classificará os diversos objetos atráves do treinamento é um dos motivos para o constante uso dessa tecnologia em diversos projetos. Devido o grande tempo levado para identificar áudios susupeitos no contexto de tráfico de droga no ambito pericial, o uso de algoritmos de rede neural convolucional pertencente ao deep learning (um dos ramos de aprendizado de máquina) mostra uma grande eficácia na identificação dessas mídias de forma automatizada atráves do treinamento de expressões espcíficos onde envolve contéudos suspeitos. 


\section{Introdução}

\section{Related Work}

\section{Metodologia}

\section{Análise e Resultados}

\section{Conclusão}

\end{document}
